\documentclass[titlepage, a4paper]{amsbook}
\usepackage[T1]{fontenc}
\usepackage[utf8, latin1]{inputenc}
\usepackage[ngerman, english]{babel}
\usepackage{amsmath}
\usepackage{amssymb}
\usepackage[T1]{fontenc}
\usepackage{eurosym}
\usepackage{selinput}
\usepackage{graphicx}
\usepackage[boxed]{algorithm2e}
\usepackage{float}
\usepackage{listings}
\usepackage{longtable}
\usepackage{array}

\linespread{1.2}
% http://www.gap-system.org/Manuals/doc/ref/chap4.html
\lstdefinelanguage{GAP}{%
  morekeywords={%
    Assert,Info,IsBound,QUIT,%
    TryNextMethod,Unbind,and,break,%
    continue,do,elif,%
    else,end,false,fi,for,%
    function,if,in,local,%
    mod,not,od,or,%
    quit,rec,repeat,return,%
    then,true,until,while%
  },%
  sensitive,%
  morecomment=[l]\#,%
  morestring=[b]",%
  morestring=[b]',%
}[keywords,comments,strings]

\newcommand{\R}{\ensuremath{\mathbb{R}}}
\newcommand{\N}{\ensuremath{\mathbb{N}}}
\usepackage[variablett]{lmodern}
\usepackage{xcolor}
\lstset{
  basicstyle=\ttfamily,
  keywordstyle=\color{red},
  stringstyle=\color{blue},
  commentstyle=\color{green!70!black},
  columns=fullflexible,
}
\newtheoremstyle{break}%
{}{}%
{}{}%
{}{}%  % Note that final punctuation is omitted.
{\newline}{}
\theoremstyle{plain}
 \newtheorem{thm}{Theorem}[chapter]
 \newtheorem{prop}[thm]{Proposition}
 \newtheorem{lem}[thm]{Lemma}
 \newtheorem{cor}[thm]{Corollary}
 \newtheorem{alg}[thm]{Algorithm}
\theoremstyle{break}
 \newtheorem{exm}[thm]{Example}
\theoremstyle{definition}
 \newtheorem{dfn}[thm]{Definition}
\theoremstyle{remark}
 \newtheorem{rem}[thm]{Remark}
 \numberwithin{equation}{thm}
 
\binoppenalty=9999
\relpenalty=9999
\DeclareMathOperator{\NP}{NP}
\begin{document}
Betrachten wir die Situation eines einfachen Knapsack Problems:
\begin{align*}
    \min\,c^Tx \\
    \text{s.t. }a^Tx \leq b.
\end{align*}
Betrachte weiterhin hiervon die robuste Formulierung mit zusaetzlich gueltiger Ungleichung
\[\sum_i \hat{c}_i d_i x_i \leq ez + \sum_i d_i p_i,\]
wobei $d_i \geq 0$. \\ Was hilft uns diese zusaetzliche Ungleichung.
\begin{lem}
Es existiert eine optimale Loesung $(x^*,z^*,p^*)$ der relaxierten Variante des obigen robusten Problems mit der zus. Ungleichung, s.d. fuer \[N:=\{i \in \{1, \ldots, n\} \mid 0< x^*_i <1, \hat{c}_i > 0 , a_i >0 \}\] gilt, dass hoechstens ein Index $j \in N$ existiert mit 
\[\hat{c}_j x^*_j < z^* \]
(in diesem Fall ist $p^*_j=0$).
\end{lem}
\begin{proof}
Sei $(x^*,z^*,p^*)$ die optimale Loesung s.d fuer
\[M_{x^*}=N \cap \{i \in \{1, \ldots, n\} \mid \hat{c}_ix^*_i < z^*\}\]
$\vert M_{x} \vert$ minimal. \\
Falls $\vert M_{x^*} \vert \geq 2$ ex. $k,l \in M_x$, s.d. $\hat{c}_kx^*_l < z^*$ und $\hat{c}_kx^*_k < z^*$. OBdA gelte $\frac{c_k}{a_k} \leq \frac{c_l}{a_l}$ (es gilt eh Gleichheit).\\
Waehle nun $y_l=x^*_l-\delta$ und $y_k=x^*_k+\frac{a_l}{a_k}\delta$
fuer $\delta \in (0, \min(\frac{a_k}{a_l}(1-x^*_k), x^*_l)]$. \\
Es gilt nun 
\[a_l y_l + a_k y_k = a_l x^*_l + a_k x^*_k\]
und 
\[c_l y_l + c_k y_k = c_l x^*_l + c_k x^*_k + \delta (\frac{c_k}{a_k} - \frac{c_l}{a_l}) \leq c_l x^*_l + c_k x^*_k .\]
Da Optimalitaet gilt muss Gleichheit gelten und es gilt $\frac{c_k}{a_k} = \frac{c_l}{a_l}$. 
Ausserdem gilt 
\[\hat{c}_ld_ly_l+\hat{c}_kd_ky_k = \hat{c}_ld_lx^*_l+\hat{c}_kd_kx^*_k + \delta (\hat{c}_k\frac{a_l}{a_k}d_k - \hat{c}_ld_l),\]
OBdA koennen wir annehmen, dass $(\hat{c}_k\frac{a_l}{a_k}d_k - \hat{c}_ld_l) \leq 0$ gilt. (d.h. $\frac{\hat{c}_k}{\hat{c}_l}\leq \frac{d_la_k}{d_ka_l}$, die Alternative ist $\frac{\hat{c}_k}{\hat{c}_l}\geq\frac{d_la_k}{d_ka_l}$ oder auch $(\hat{c}_l\frac{a_k}{a_l}d_l - \hat{c}_kd_k) \leq 0$).Damit bleibt \[\sum_i \hat{c}_i d_i x_i \leq ez + \sum_i d_i p_i\] gueltig fuer die Verschiebung. \\
Dann kann aber so viel $\delta$ verschoben werden, wie moeglich, d.h.
$\delta = \min(\frac{a_k}{a_l}(1-x^*_k), x^*_l)$. Dann gilt aber $y_l=0$ oder $y_k=1$. Fuer $y_i=x^*_i$ sonst, ist dann aber $y$ eine optimale Loesung mit $\vert M_y \vert < \vert M_{x^*} \vert$, was ein Widerspruch ist. \\
Also muss am Anfang bereits $M_{x^*}$ einelementig oder weniger gewesen sein.
\end{proof}
Also kann fuer eine optimale Loesung des relaxierten robusten Problems mit zusaetzlicher Ungleichung davon ausgegangen werden, dass hoechstens ein fraktionaler Eintrag $x^*_j$ (abgesehen von Eintraegen mit $\hat{c}_i=0=a_i$) der nicht Gleichheit erfuellt ( $\hat{c}_nx^*_n = z^* + p^*_n$ fuer alle $n \in N \setminus\{j\}$).
Hiervon koennen wir ausgehend von der Ungleichung 
\[\sum_i \hat{c}_i d_i x_i \leq ez + \sum_i d_i p_i\]
eine Schranke an solche fraktionalen Eintraege geben, damit diese Ungleichung noch erfuellt ist.
Betrachte hierfuer 
\[Eq=\{i \in \{1, \ldots, n\} \mid \hat{c}_i x^*_i = z^* + p^*_i\}.\]
Dann ist
\[\sum_{i \in Eq}\hat{c}_i d_i x^*_i = \sum_{i \in Eq}z^* d_i + \sum_{i \in Eq}p^*_i d_i = \sum_{i \in Eq}z^* d_i + \sum_{i \in \{1, \ldots, n\}}p^*_i d_i\]
da $p^*_k=0$, fuer $k \notin Eq$.
Da
\[\sum_{i \in Eq}z^* d_i + \sum_{i \in \{1, \ldots, n\}}p^*_i d_i= z^*(\sum_{i \in Eq\setminus N} d_i+ \sum_{i\in N\setminus\{j\}} d_i)+ \sum_{i \in \{1, \ldots, n\}}p^*_i d_i\]
ist fuer
\[z^*(\sum_{i \in Eq\setminus N} d_i+ \sum_{i \in N\setminus\{j\}} d_i)+ \sum_{i \in \{1, \ldots, n\}}p^*_i d_i > z^* e + \sum_{i \in \{1, \ldots, n\}}p^*_i d_i\]
sprich fuer
\[z^*(\sum_{i \in Eq\setminus N} d_i+ \sum_{i \in N\setminus\{j\}} d_i) > z^* e\]
die zusaetzliche Ungleichung ungueltig.
Falls wir nun OBdA annehmen, dass $d_1 \leq d_2 \leq \ldots \leq d_n$ gilt, und da insbesondere $\sum_{i \in N\setminus\{j\}} d_i > e$ fuer eine gueltige Loesung nicht erlaubt ist, so gilt also insbesondere fuer eine gueltige Loesung $(x^*,z^*,p^*)$ des robusten relaxierten Problems \textbf{mit} zusaetzlicher Ungleichung, dass fuer 
\[m = \min\{k \mid \sum_{i=1}^k d_i > e\}\]
gilt $m+1 > \vert N \vert$.\\
Also wird die Anzahl der fraktionalen Koeffizienten der Relaxierung \textbf{mit} der zusaetzlichen Ungleichung beschraenkt durch Eigenschaften der Ungleichung. \\
Insbesondere ist dies nuetzlich bei verallgemeinerten Cliquen-Ungleichungen
\[\sum_{i=m}^n x_i \leq k\]
fuer $k\geq 1$. In einem Knapsack Problem mit dieser gueltigen Ungleichung koennen in der Variablenmenge $\{x_m, \ldots, x_n\}$ (falls jewweils $a_i,\hat{c}_i > 0$ gilt) hoechstens $k$ Variablen fraktional sein.
\end{document}